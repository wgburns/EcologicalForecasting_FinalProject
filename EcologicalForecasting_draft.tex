\documentclass[]{article}
\usepackage{lmodern}
\usepackage{amssymb,amsmath}
\usepackage{ifxetex,ifluatex}
\usepackage{fixltx2e} % provides \textsubscript
\ifnum 0\ifxetex 1\fi\ifluatex 1\fi=0 % if pdftex
  \usepackage[T1]{fontenc}
  \usepackage[utf8]{inputenc}
\else % if luatex or xelatex
  \ifxetex
    \usepackage{mathspec}
  \else
    \usepackage{fontspec}
  \fi
  \defaultfontfeatures{Ligatures=TeX,Scale=MatchLowercase}
\fi
% use upquote if available, for straight quotes in verbatim environments
\IfFileExists{upquote.sty}{\usepackage{upquote}}{}
% use microtype if available
\IfFileExists{microtype.sty}{%
\usepackage{microtype}
\UseMicrotypeSet[protrusion]{basicmath} % disable protrusion for tt fonts
}{}
\usepackage[margin=1in]{geometry}
\usepackage{hyperref}
\hypersetup{unicode=true,
            pdftitle={Ecological Forecasting Final Project Draft},
            pdfauthor={WB},
            pdfborder={0 0 0},
            breaklinks=true}
\urlstyle{same}  % don't use monospace font for urls
\usepackage{natbib}
\bibliographystyle{plainnat}
\usepackage{graphicx,grffile}
\makeatletter
\def\maxwidth{\ifdim\Gin@nat@width>\linewidth\linewidth\else\Gin@nat@width\fi}
\def\maxheight{\ifdim\Gin@nat@height>\textheight\textheight\else\Gin@nat@height\fi}
\makeatother
% Scale images if necessary, so that they will not overflow the page
% margins by default, and it is still possible to overwrite the defaults
% using explicit options in \includegraphics[width, height, ...]{}
\setkeys{Gin}{width=\maxwidth,height=\maxheight,keepaspectratio}
\IfFileExists{parskip.sty}{%
\usepackage{parskip}
}{% else
\setlength{\parindent}{0pt}
\setlength{\parskip}{6pt plus 2pt minus 1pt}
}
\setlength{\emergencystretch}{3em}  % prevent overfull lines
\providecommand{\tightlist}{%
  \setlength{\itemsep}{0pt}\setlength{\parskip}{0pt}}
\setcounter{secnumdepth}{0}
% Redefines (sub)paragraphs to behave more like sections
\ifx\paragraph\undefined\else
\let\oldparagraph\paragraph
\renewcommand{\paragraph}[1]{\oldparagraph{#1}\mbox{}}
\fi
\ifx\subparagraph\undefined\else
\let\oldsubparagraph\subparagraph
\renewcommand{\subparagraph}[1]{\oldsubparagraph{#1}\mbox{}}
\fi

%%% Use protect on footnotes to avoid problems with footnotes in titles
\let\rmarkdownfootnote\footnote%
\def\footnote{\protect\rmarkdownfootnote}

%%% Change title format to be more compact
\usepackage{titling}

% Create subtitle command for use in maketitle
\providecommand{\subtitle}[1]{
  \posttitle{
    \begin{center}\large#1\end{center}
    }
}

\setlength{\droptitle}{-2em}

  \title{Ecological Forecasting Final Project Draft}
    \pretitle{\vspace{\droptitle}\centering\huge}
  \posttitle{\par}
    \author{WB}
    \preauthor{\centering\large\emph}
  \postauthor{\par}
      \predate{\centering\large\emph}
  \postdate{\par}
    \date{12/2/2019}


\begin{document}
\maketitle

\hypertarget{introduction-wb}{%
\subsection{1. Introduction (WB)}\label{introduction-wb}}

GENERAL NOTES:

\begin{itemize}
\tightlist
\item
  why important to study?
\item
  cite a few other studies doing similar work
\item
  end intro with ``in this study'' paragraph to set up reader to how we
  plan to attack the problem
\item
  hypotheses?
\item
  Figs for this section:

  \begin{itemize}
  \tightlist
  \item
    map of study area
  \item
    bar graph of cyanobacteria bloom timing being different every year
    (long-term monitoring)
  \end{itemize}
\end{itemize}

Anthropogenic eutrophication, meaning increases in nutrients entering
aquatic systems due to human activity, of natural water bodies is a
global phenomenon that often leads to decreased utility of affected
ecosystems (eg. Reynolds 1987; Paerl 1988; Paerl et al.~2011). One
problem associated with eutrophication is that harmful cyanobacteria
blooms are becoming more prevalent in coastal oceans and thousands of
inland lakes and ponds (Paerl et al.~2001, 2011; Paerl and Huisman
2008). Phytoplankton play an integral role as the base of aquatic food
webs in both freshwater and marine systems, however, when cyanobacteria
growth is unchecked, blooms can be detrimental to organisms living in
and around freshwater ecosystems and often cause problems that propagate
up the food web (add REFS). The climate is changing, and environmental
conditions are becoming more favorable for bloom-forming harmful
cyanobacteria (Jöhnk et al.~2006; Paerl and Huisman 2008; Jeppesen et
al.~2011; Huisman et al.~2018). Therefore, it is becoming increasingly
important to understand cyanobacteria bloom onset and senescence.
Understanding why and when severe cyanobacteria blooms occur will help
inform mitigation efforts.

\hypertarget{factors-that-drive-cyanobacteria-blooms}{%
\subsubsection{1.1 Factors that drive cyanobacteria
blooms}\label{factors-that-drive-cyanobacteria-blooms}}

\textbf{Goal with this section: Develop what is known from what is not
known or what has conflicting attributions (and why), and how this leads
you to your research question. This driver -- that is, different
watersheds and connectivity -- needs to be developed in the Intro to
show the reader this is a valid research question/hypothesis to pursue.}

\begin{itemize}
\tightlist
\item
  Eutrophication (can mention watershed bay interactions in this
  section)
\item
  Warm temperatures
\item
  Water column stratification (leads to internal loading of nutrients)
\item
  Increased CO2 in the atmosphere (some types of cyanobacteria more
  efficiently produce organic matter
\item
  Rain and wind events (mix up water column prohibiting internal loading
  of nutrients and also giving advantage to larger phytoplankton, like
  diatoms, that need turbulence to get mixed up into the photic zone)
\item
  Cyanobacteria are physiologically diverse and have developed multiple
  strategies to out compete other types of phytoplankton: N-fixation,
  CO2 concentrating mechanisms, buoyancy regulation, toxin production,
  predator avoidance
\item
  Top-down grazing by zooplankton -- however, many studies have shown
  that cyanobacteria often avoid grazing by forming dense colonies and
  toxin production (DeMott 1986; Lemaire et al.~2012)
\item
  Viral lysis or fungal infection -- often don't result in long-lasting
  effects on cyanobacteria populations (Yoshida et al.~2008; Van
  Wichelen et al.~2016)
\item
  Filter feeders like mussels - effect they have on cyanobacteria blooms
  is lake-specific (Reeders et al.~1989; Vanderploeg et al.~2001)
\item
  Competition from other, non-harmful, phytoplankton
\end{itemize}

How does lake size (area/depth) affect which drivers may be relatively
important (shallow vs deep)? Lake trophic status?

\hypertarget{methods}{%
\subsection{2. Methods}\label{methods}}

\begin{itemize}
\tightlist
\item
  study sites (WB)
\item
  data used and how prepped

  \begin{itemize}
  \tightlist
  \item
    buoy sensor data (MC)
  \item
    meteorological station data (WB): air temp, solar radiation, wind
    speed, wind direction (WB)
  \item
    discharge data (WB)
  \end{itemize}
\item
  data analysis - feature important
\end{itemize}

Feature importance Is a way of understanding the factors that most
contribute to the mechanisms of a system. In our studty we used XGBOOST
a machine learning technique that has been shown to be most accurate in
tabular and structured data. It replaces Multiple Additive Regression
Trees(MARTs) and uses gradient boosted decision trees and is built for
speed and perfomance and hence is more accurate than alternatives.
\includegraphics{"/Users/chege/Desktop/EcologicalForecasting_FinalProject/SpeedFigure.png"}

\hypertarget{preparing-the-high-frequency-buoy-data-mc}{%
\subsubsection{2.2.1 Preparing the high frequency buoy data
(MC)}\label{preparing-the-high-frequency-buoy-data-mc}}

High frequency water quality data was collected from May through October
in 2017 and 2018 by sensors on two buoys in Lake Champlain: one in
St.~Albans Bay, and one in Missisquoi Bay. Each buoy had sensors at
multiple depths, placed every 0.5m from a depth of 0.5m to the bottom
(2.5m depth in Missisquoi Bay, 4.5m depth in St.~Albans Bay). Sensors
measured temperature, conductivity, pH, dissolved oxygen, chlorophyll
(Chl), phycocyanin (PC), and turbidity. DeltaTemp was also calculated
for each time point as a measure of lake stratification by subtracting
the temperature at the bottom from that at the surface.

Since the buoys collect data at multiple depths, we first considered
whether to focus exclusively on the sensors nearest the surface, or to
aggregate data from whichever depth had the highest concentration of PC
at a given time, so as to track a bloom as it moves up and down in the
water column. For each bay and year, we explored how PC values varied
with depth. First, for each bay and year, we used R to create a
correlation matrix with Pearson correlation coefficients for the PC
levels at each depth. We found PC levels across depths to be moderately
to highly correlated in each case (0.38-1.00), indicating that PC levels
near the surface would most likely be representative of those throughout
the water column. We also used R to compare the PC levels across depths
at each time point, and find out at what depth the PC level was
maximized. We found that for periods of low PC levels, the maximum might
be found at any depth, but for periods of high PC levels, such as during
a cyanobacteria bloom, maximum PC levels where most often found near the
surface. For these reasons, we decided to limit ourselves to the data
collected by sensors nearest the surface (0.5m depth), augmented with
the deltaTemp as a measure of stratification.

In order to run a forecasting model or look at feature importance with
time lags, we needed to collapse this hourly data into daily data, so
that daily cycles would not confound predictions or time lags. We
explored whether to use daily averages or daily maximums. R was used to
calculate daily averages and maximums for each buoy variable, and to
create time plots of the daily and hourly data. After inspecting the
time plots, we decided to use daily averages rather than daily maximums,
as the later were unduly influenced by sudden brief peaks or high-valued
outliers in the hourly data.

In addition to collapsing our hourly buoy data into daily averages for
each environmental variable, we wanted to explore how water quality data
related to PC values over different time lags. In order to include this
lag, we used R to append the daily average PC levels 1-6 days in the
future to the data for each time step.

\hypertarget{results-wb}{%
\subsection{3. Results (WB)}\label{results-wb}}

\begin{itemize}
\tightlist
\item
  fig of time series of the parameters to set the scene?
\item
  which set of feature importance do we want to use?
\end{itemize}

Next steps for today 12/2/19: - figure out way to make figure that is
easy to digest --\textgreater{} take out Chl, pH, and ODO and then have
a stacked bar graph with y axis of feature importance (all add up to 1)
and x axis being days in future (0 - 6) - need to sep feature importance
by bay and year AND figure out how to make a table with the results so i
can make the stacked bar graph

\hypertarget{brief-typed-out-results-for-easier-interpretation-not-going-to-include}{%
\paragraph{Brief typed out results for easier interpretation (NOT GOING
TO
INCLUDE)}\label{brief-typed-out-results-for-easier-interpretation-not-going-to-include}}

\begin{enumerate}
\def\labelenumi{\arabic{enumi})}
\tightlist
\item
  St AB both years
\end{enumerate}

\begin{itemize}
\tightlist
\item
  0 day lag: pH, spCond, temp, discharge, deltaTemp
\item
  1 day lag: pH, spCond, discharge, temp
\item
  2 day lag: pH, spCond, discharge, ODO
\item
  3 day lag: pH, spCond, discharge, deltaTemp
\item
  4 day lag: pH, spCond, deltaTemp, ws
\item
  5 day lag: pH, spCond, deltaTemp, wd, ODO
\item
  6 day lag: pH, spCond, ODO, ws, discharge
\end{itemize}

\begin{enumerate}
\def\labelenumi{\arabic{enumi})}
\setcounter{enumi}{1}
\tightlist
\item
  MB both years
\end{enumerate}

\begin{itemize}
\tightlist
\item
  0 day lag: pH, ODO, discharge, wd (deltaTemp last)
\item
  1 day lag: pH, discharge, ODO, spCond, wd (deltaTemp moves up)
\item
  2 day lag: pH, ODO, wd, solar, spCond,
\item
  3 day lag: pH, spCond, temp, wd, deltaTemp
\item
  4 day lag: spCond, pH, temp, ODO, ws, deltaTemp
\item
  5 day lag: spCond, temp, pH, ODO, deltaTemp, wd
\item
  6 day lag: spCond, temp, pH, ODO, wd, discharge, deltaTemp
\end{itemize}

\begin{enumerate}
\def\labelenumi{\arabic{enumi})}
\setcounter{enumi}{2}
\tightlist
\item
  2017 both bays
\end{enumerate}

\begin{itemize}
\tightlist
\item
  0 day lag: discharge, spCond, pH, ODO
\item
  1 day lag: spCond, ODO, discharge, pH, solar
\item
  2 day lag: pH, spCond, discharge, ws, ODO
\item
  3 day lag: spCond, ODO, discharge, solar
\item
  4 day lag: spCond, ODO, solar, wd, temp
\item
  5 day lag: spCond, deltaTemp, solar, wd, temp
\item
  6 day lag: deltaTemp, spCond, ODO, wd, discharge
\end{itemize}

\begin{enumerate}
\def\labelenumi{\arabic{enumi})}
\setcounter{enumi}{3}
\tightlist
\item
  2018 both bays
\end{enumerate}

\begin{itemize}
\tightlist
\item
  0 day lag: pH, spCond, ODO, discharge
\item
  1 day lag: pH, spCond, discharge, ODO
\item
  2 day lag: pH, discharge, temp, spCond
\item
  3 day lag: pH, ODO, discharge, spCond, temp
\item
  4 day lag: pH, spCond, discharge, temp, deltaTemp
\item
  5 day lag: pH, spCond, temp, discharge, ODO
\item
  6 day lag: pH, spCond, temp, solar, ODO
\end{itemize}

\begin{enumerate}
\def\labelenumi{\arabic{enumi})}
\setcounter{enumi}{4}
\tightlist
\item
  both bays both years
\end{enumerate}

\begin{itemize}
\tightlist
\item
  0 day lag: pH, spCond, year, discharge, solar
\item
  1 day lag: pH, spCond, year, discharge, temp
\item
  2 day lag: pH, spCond, discharge, year, deltaTemp
\item
  3 day lag: pH, spCond, discharge, year, ODO
\item
  4 day lag: pH, spCond, year, discharge, ODO
\item
  5 day lag: pH, spCond, temp, ODO, year, discharge
\item
  6 day lag: pH, spCond, temp, year, deltaTemp, ODO
\end{itemize}

LOOK INTO - is spCond just super highly correlated with PC? Why is it
such an important feature? What is it highly correlated with? are the
units different between the bays? - wd (not speed) is important for MB!
let's look at correlations - should we run without pH? - what's the
significance in the difference in the cluster numbers??

\hypertarget{discussion}{%
\subsection{Discussion}\label{discussion}}

\begin{itemize}
\tightlist
\item
  what we know right now is that 6 days out, pH is IMPORTANT for PC but
  the next step is figuring out if it's PREDICTIVE of a bloom
  --\textgreater{} need to build model
\item
  We got kinda hung up on the fact that there were parameters in the
  feature analysis results that were highly correlated with PC (response
  variable) but Easton reminded us that if our main goal is to predict a
  bloom, Chl or PC levels 6 days before is important to include if we
  are building a predictive model because that's still useful info !
  Like it'd be great if we could predict if a bloom would start in 6
  days just by knowing PC or Chl levels.
\item
  IMPORTANT POINT: there is a difference between mechanistic models and
  predictive models. Depending on how we decide to move forward (ie.
  which model we choose) we might not really need to understand the
  mechanisms driving the blooms because we might just try throwing all
  the data into a machine learning algorithm that tells us the relative
  predictive power of each of the parameters (I think\ldots).
\end{itemize}

\hypertarget{future-work-mc}{%
\subsection{Future Work (MC)}\label{future-work-mc}}

We could build on this work in a number of ways, by expanding our data
set, correlating our phycocyanin measurements with satellite data or
volunteer observations, identifying a bloom threshold, and most
importantly, by using our high frequency data to create a forecasting
model that could predict cyanobacteria blooms.

First, we could expand our data set by including cumulative degree days
as another environmental variable, indicative of temperatures
experienced throughout the season up to a given time point. We could
collect mean daily temperatures for 2017 and 2018 for the nearest
weather stations to the two bays from wunderground.com, and calculate
the cumulative degrees above freezing (or above a biologically relevant
temperature threshold such as 4C) for each day. We could then include
this variable in analyses such as correlations, feature importance, or a
forecasting model.

Second, we could correlate the buoys' measure of PC levels with other
indicators of bloom presence such as volunteer observations and
satellite data. There is an online data set publicly available with
volunteer observations at the two bays: biweekly observations of bloom
presence or absence throughout the 2017 and 2018 seasons. Comparing
these observations with the daily average PC levels from our buoy
sensors on the days of the observations could help us identify a
threshold for what PC levels indicate a cyanobacteria bloom for
management purposes.

There is also satellite imagery available covering both bays, and it can
be used to calculate a spectral index that's indicative of cyanobacteria
presence. We could calculate this index at the position of each buoy for
the time points of the available satellite imagery. We could then see
how the index correlates with the daily average PC levels measured by
the buoys, and examine whether there is a simple, consistent conversion
between the two. If there are literature thresholds for what value of
the spectral index constitutes a cyanobacteria bloom, we could then use
a correlation or conversion to calculate an analogous bloom threshold
for our buoy-measured PC levels. Having an accurate bloom threshold
would be valuable for investigating bloom drivers and creating a
forecasting model. It would give us the option to investigate continuous
PC values directly, or convert them to categorical `Bloom'/`No Bloom'
data in case the later is easier to forecast or more strongly correlated
with certain drivers.

Finally, our long term goal is to use this high frequency data to
develop a forecasting model using machine learning techniques, in the
hopes of predicting cyanobacteria blooms a few days before they occur.
We would begin by inputting all available high frequency buoy, weather
and discharge data, in order to predict PC levels (or a categorical
variable for bloom presence or absence). If that proved successful, we
would then begin to remove input variables one by one, to see what
effect that has on the forecast accuracy. The goal would be to have a
functional forecasting model with as few input variables as possible.

\hypertarget{references}{%
\subsection{References}\label{references}}

INFO ON HOW TO CITE in .Rmd from
\url{https://rmarkdown.rstudio.com/authoring_bibliographies_and_citations.html}
:

\begin{itemize}
\tightlist
\item
  Citations go inside square brackets and are separated by semicolons.
  Each citation must have a key, composed of `@' + the citation
  identifier from the database, and may optionally have a prefix, a
  locator, and a suffix. Here are some examples:

  \begin{itemize}
  \tightlist
  \item
    Blah blah \citep{isles_modeling_2017}.
  \end{itemize}
\item
  Then in Zotero create a \texttt{.BibTex} file by going to File
  --\textgreater{} Export Library --\textgreater{} change to BibTex
\end{itemize}

\bibliography{EcologicalForecastingRefs.bib}


\end{document}
